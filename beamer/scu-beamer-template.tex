\documentclass{beamer}
\mode<presentation>
\usepackage{ctex}
\usepackage{calligra}
\usepackage[T1]{fontenc}

<<<<<<< HEAD
\usepackage{multimedia}
%% \usepackage{ctex}

\title{ \title{There Is No Largest Prime Number}}
\date[ISPN ’80]{27th International Symposium of Prime Numbers}
\author[Euclid]{Euclid of Alexandria \\ \texttt{euclid@alexandria.edu}}
=======
\usetheme{Darmstadt}
\title{中国特检院求职陈述}
\date{\today}
\author{王逸舟 \\ \texttt{mathwyz@163.com}}
\institute[josephpan]{ \includegraphics[height=1cm]{png/yuanxun.jpg}\\
  四川大学数学学院
 }
>>>>>>> cf814d3c7a746f2fe9ff3a7b5208f1be81f0a350

\logo{\includegraphics[scale=0.2]{png/xiaohui.png}}


\begin{document}
\begin{frame}
\titlepage
\end{frame}


% \begin{frame}
% \frametitle{Outline}
% \tableofcontents[pausesections]
% \end{frame}

\begin{frame}
  \frametitle{基本信息}
  \begin{itemize}
  \item 姓名 :王逸舟
  \item 出生年月 :1997.12.11
  \item 籍贯和户口 :甘肃省天水市
  \item 学历:
    \begin{itemize}
    \item 本科 四川大学 统计学 2014.09-2018.06
      \begin{itemize}
      \item 数学分析 高等代数 解析几何
      \item 初等数论 抽象代数 复变函数
      \item 实变函数 泛函分析 微分方程
      \item 线性模型 随机过程 多元统计
      \item 时间序列 统计计算 实验设计
      \end{itemize}
    \item 硕士 四川大学 数学 2018.09-2021.06
      \begin{itemize}
      \item 微分方程定性理论 动力系统基础 向量场的分岔理论
      \item 泛函分析 微分方程 函数方程
      \item 微分方程周期解 
      \end{itemize}
    \end{itemize}
  \end{itemize}
\end{frame}


\begin{frame}
  \frametitle{计算机技能}
  \begin{itemize}
  \item 编程语言:C(Linux C 基本开发,熟悉Emacs,gcc,gdb),Python
  \item 开发环境:熟悉Linux基本运维管理和熟悉shell编程;熟练使用Emacs,Vim,熟练使用git和github; 熟练使用docker
  \item 数学建模:熟悉微分方程建模,微分方程组建模,熟悉科学计算软件Matlab,Octave,Sage,Maple,Maxima,Mathematics,LaTeX
  \item 统计分析:熟悉回归分析,时间序列熟悉统计软件R,SAS,Python
  \end{itemize}
\end{frame}

\begin{frame}
  \frametitle{项目}
 \begin{itemize}
\item 《向量场的分岔理论》排版 排版页数80多页 代码7820行
\pause
\item 统计学作业 代码量2176行 
\pause
\item wuhan2020 更新博客23个
  
\end{itemize}

\end{frame}

\begin{frame}
  \frametitle{论文}
  
  \begin{itemize}
  \item Polynomial differential systems without closed orbits
    \pause
  \item 线性系统解的代数结构
    \pause
  \item 多元函数的任意阶方向导数
    \pause
  \item 一类微生物模型的定性分析
    \pause
  \end{itemize}
  
\end{frame}

\begin{frame}
  \frametitle{暑期项目}
  \begin{itemize}
  \item 中科院开源软件点亮计划(春松客服) 9000.00元
    \pause
  \item 清华大学猛士无人驾驶 100.00元/天
    \pause
  \end{itemize}
\end{frame}

\begin{frame}
  \begin{center}
\Huge { 谢谢老师!祝我顺利!}
  \end{center}
\end{frame}

<<<<<<< HEAD
\begin{frame}
\sound[autostart,samplingrate=705000,bitspersample=16,
channels=2]{Example}{notify.wav}
\end{frame}

=======
>>>>>>> cf814d3c7a746f2fe9ff3a7b5208f1be81f0a350
\end{document}
