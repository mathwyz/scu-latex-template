\documentclass{beamer}
\mode<presentation>

\usepackage{multimedia}
%% \usepackage{ctex}

\title{ \title{There Is No Largest Prime Number}}
\date[ISPN ’80]{27th International Symposium of Prime Numbers}
\author[Euclid]{Euclid of Alexandria \\ \texttt{euclid@alexandria.edu}}

\logo{\includegraphics{png/xiaohui.png}}


\begin{document}
\begin{frame}
\titlepage
\end{frame}


\begin{frame}
\frametitle{Outline}
\tableofcontents[pausesections]
\end{frame}

\begin{itemize}
\item 2 is prime (two divisors: 1 and 2).
\pause
\item 3 is prime (two divisors: 1 and 3).
\pause
\item 4 is not prime (\alert{three} divisors: 1, 2, and 4).
\end{itemize}



\begin{frame}
\frametitle{There Is No Largest Prime Number}
\framesubtitle{The proof uses \textit{reductio ad absurdum}.}
\begin{theorem}
There is no largest prime number.
\end{theorem}
\begin{proof}
\begin{enumerate}
\item<1-| alert@1> Suppose $p$ were the largest prime number.
\item<2-> Let $q$ be the product of the first $p$ numbers.
\item<3-> Then $q+1$ is not divisible by any of them.
\item<1-> But $q + 1$ is greater than $1$, thus divisible by some prime
number not in the first $p$ numbers.\qedhere
\end{enumerate}
\end{proof}
\end{frame}

\begin{frame}
\frametitle{User guide}

\begin{block}{User guide}
First slider
\end{block}


\end{frame}


\begin{frame}
  \begin{block}{Second}
Second slider
    \end{block}
  \end{frame}


\begin{frame}
Thanks!
\end{frame}


\begin{frame}
\color<2-3>[rgb]{1,0,0} This text is red on slides 2 and 3, otherwise black.
\end{frame}

\begin{frame}
Shown on first slide.
\onslide<2-3>
Shown on second and third slide.
\begin{itemize}
\item
Still shown on the second and third slide.
\onslide+<4->
\item
Shown from slide 4 on.
\end{itemize}
Shown from slide 4 on.
\onslide
Shown on all slides.
\end{frame}

\begin{frame}
\sound[autostart,samplingrate=705000,bitspersample=16,
channels=2]{Example}{notify.wav}
\end{frame}

\end{document}
